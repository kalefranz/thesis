\documentclass[a4paper,10pt]{report}
\usepackage{amsmath}
\usepackage{amsfonts}
\usepackage{amssymb}

% Title Page
\title{Phys 261}
\author{The class}


\begin{document}
\maketitle

\begin{abstract}
Here are the notes from the class. They are written i realtime.
\end{abstract}
\section{Third lecture 29.08.06}
\subsection{Atomic world construction}
$a_0$ - radius, length
$\Delta x \approx a_0$ $\Delta k \approx 1/a_0$
($\hbar \Delta k = \Delta p$ wavenumber
$\Delta E \approx \frac{(\Delta p )^2}{2m} = \frac{\hbar ^2}{2ma_0^2} \approx \frac{\hbar ^2}{ma_0^2}$ (kinetic energy T$_0$. )
Potential energy V$_0$
$|-\frac{e^2}{a_0}$
$a_0 = 0,529$ \AA = $\frac{\hbar^2}{e^2m}$ Lenght unit
Energy unit $\frac{e^2}{a_0} = \frac{\hbar}{ma_0^2} = 27,2$ eV

$<T_0>$ = $<V_0>$ should be in the same order.

Atomic unit of time
Atomic unit of velocity $p_0 = \hbar k_0 = \frac{\hbar}{a_0}
v_0 = \frac{p_0}{m} = \frac{\hbar}{m}\frac{1}{a_0} = \frac{me^2}{\hbar ^2}\frac{\hbar}{m} = \frac{e^2}{\hbar}
\frac{v_0}{c} = \frac{e^2}{\hbar c} = \alpha = \frac{1}{137}$
$t_0 =\frac{a_0}{v_0} = (\frac{a_0}{e^2}) \hbar = \frac{\hbar}{E_0}$
Alternative postulate $t_0 = \frac{\hbar}{E-O}$

(Statement a.u. $\longleftrightarrow e = m \hbar = 1$)

$\hbar = 0,66 \cdot 10^{-15}$ eVs $t_0 = \frac{0,66 \cdot 10^{-15}}{27,2}$s=$0,24 \cdot 10^{-15}$
 (2 $\pi$ for angular freq.)
$\nu$ frequency | $\omega$ - angular frequency

$k_0 T$ is the ``physical temperature''.
Room temperature is thus $\frac{1}{40}$eV or 25 meV

Atomic unit of energy $\longrightarrow$ VERY HOT

More about bound states in H

Ground state $-\frac{1}{2}$ a.u.
States characterized by $n, l$ ($m$ - magnetic)
In H energies given by $\frac{1}{n^2}(-\frac{1}{2})$a.u
 **Sett inn bilde av matrise og brønnpotensiale har bilder**

Ladi says its nonsens to talk about m as a quantum number.

Angular momentum
$L = \omega \varphi$ $T_{rot} = \frac{1}{2} \frac{L^2}{g}$
$E=t+V$ is negativ

 ** bane bilde **

$I_n$ QM it looks different
3-dim Schr.Eq. $\rightarrow$ Seperation of variables
$x, y, z, \rightarrow r, \nu , \varphi | \frac{\delta ^2}{\delta x^2}+\frac{\delta ^2}{\delta y^2}+\frac{\delta ^2}{\delta z^2} \longrightarrow T_r +\frac{L^2(\nu , \varphi )}{r^2}$

$T_r \longrightarrow \frac{1}{r}\frac{\delta}{\delta r}\frac{1}{r}\frac{\delta}{\delta r}$ $L^2$ is ugly (can be made very elegant)
This is generally used in many fields.

Exercise:
Look on the separation and how it's done.

(0)s-states, (1)p-states, (2)d-states, (3)f-states, ... there are more, but it is irrelevant here



\end{document}
