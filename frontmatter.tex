\addtolength{\textheight}{0.6in}

%%
%% COVER PAGE
%%
\thispagestyle{empty}
\vspace*{-0.6in}
\begin{center}
  \includegraphics[width=0.5in]{PrincetonShield}\\
  \vspace{0.2in}
  \huge \textsc{\textbf{New Quantum Cascade Laser\\Architectures}}\\
  \vspace{0.2in}
  \textsc{\Large II--VI Quantum Cascade Emitters,\\High \emph{k}-Space Lasing,\\
  \vspace{-0.12in}
  \& Short Injectors}
\end{center}
\vspace{.4in}
\begin{center}
  \Large Kale J. Franz
\end{center}
\vspace{.4in}
\begin{center}
\sc
    A Dissertation \\
    Presented to the Faculty \\
    of Princeton University \\
    in Candidacy for the Degree \\
    of Doctor of Philosophy
\end{center}
\vspace{.3in}
\begin{center}
\textsc{
    Recommended for Acceptance \\
    by the Department of Electrical Engineering}\\
    \vspace*{0.1in}
    Adviser: Claire Gmachl
\end{center}
\vspace*{0.9in}
\begin{center}
    June 2009
\end{center}
\clearpage


%%%
%%% COVER PAGE #2
%%%
%\thispagestyle{empty}
%\vspace*{-0.6in}
%\begin{center}
%  \vspace{0.2in}
%  \huge \textsc{\textbf{New Quantum Cascade Laser\\Architectures}}\\
%  \vspace{0.2in}
%  \textsc{\Large II--VI Quantum Cascade Emitters\\High \emph{k}-Space Lasing\\
%  \vspace{-0.12in}
%  and Short Injectors}
%\end{center}
%\vspace{.4in}
%\begin{center}
%  \Large Kale J. Franz
%\end{center}
%\vspace{.4in}
%\begin{center}
%\sc
%    A Dissertation \\
%    Presented to the Faculty \\
%    of Princeton University \\
%    in Candidacy for the Degree \\
%    of Doctor of Philosophy
%\end{center}
%\vspace{.3in}
%\begin{center}
%\textsc{
%    Recommended for Acceptance \\
%    by the Department of Electrical Engineering}\\
%    \vspace*{0.1in}
%    Adviser: Claire Gmachl
%\end{center}
%\vspace*{0.9in}
%\begin{center}
%    June 2009
%\end{center}
%\clearpage


\cleardoublepage

%%
%% COPYRIGHT PAGE
%%
\thispagestyle{empty}
\vspace*{\stretch{1}}
\begin{center}
    \copyright \ Copyright by Kale J. Franz, June 2009. \\
    All Rights Reserved
\end{center}
\pagebreak
\clearpage

\addtolength{\textheight}{-0.6in}

\cleardoublepage

%%%
%%% SIGNATURE PAGE
%%%
%\thispagestyle{empty}
%    \vspace*{-0.7in}
%    \signature{Claire Gmachl\\\textit{\small Principal Adviser}}
%    \vfill
%    \signature{Andrew A. Houck}
%    \vfill
%    \signature{Gerard Wysocki}
%    \vfill
%    \begin{center}
%    \begin{minipage}{4in}
%    \raggedleft Approved for the Princeton University Graduate School.\par
%    \vspace{.5in}
%    \hbox to 4in{\hfil\shortstack{\vrule width 3in height 0.4pt\\
%            Dean of the Graduate School}}
%    \end{minipage}
%    \vspace*{0.4in}
%    \end{center}
%\clearpage
%
%\cleardoublepage

\renewcommand{\voffset}{-14pt}
\renewcommand{\voffset}{-0.25in}
\renewcommand{\headsep}{0.35in}
\renewcommand{\textheight}{8.2in}

%%
%% ABSTRACT
%%
\newpage
\addcontentsline{toc}{section}{Abstract}
\begin{center}
  \Large \textbf{Abstract}
\end{center}

\onehalfspacing

Quantum cascade (QC) lasers are today's most capable mid-infrared light sources.  With up to watt-level room temperature emission over a broad swath of mid-infrared wavelengths, these tiny semiconductor devices enable a variety of applications and technologies such as ultra-sensitive systems for detecting trace molecules in the vapor phase.  The foundation of a QC structure lies in alternating hundreds of wide- and narrow-bandgap semiconductor layers to form a coupled quantum well system.  In this way, the laws of quantum mechanics are used to precisely engineer electron transport and create artificial optical transitions.  The result is a material with capabilities not found in nature, a truly "designer" material.

As a central theme in this thesis, we stress the remarkable flexibility of the quantum cascade---the ability to highly tailor device structure for creative design concepts.  The QC idea, in fact, relies on no particular material system for its implementation.  While all QC lasers to date have been fabricated from III--V materials such as InGaAs/AlInAs, I detail our preliminary work on ZnCdSe/ZnCdMgSe---a II--VI materials system---where we have demonstrated electroluminescence.

We then further discuss how the inherent QC flexibility can be exploited for new devices that extend QC performance and capabilities.  In this regard, we offer the examples of excited state transitions and short injectors.  Excited state transitions are an avenue to enhancing optical gain, which is especially needed for longer-wavelength devices where optical losses hinder performance.  Likewise, shortening the QC injector length over a conventional QC structure has powerful implications for threshold current, output power, and wall-plug efficiency.  In both cases, novel physical effects are discovered.  Pumping electrons into highly excited states led to the discovery of high \emph{k}-space lasing from highly non-equilibrium electron distributions.  Shortening QC injector regions allowed us to observe ``classical'' superlattice effects such as negative differential resistance and pulse instabilities.  While interesting from a scientific perspective, these unique phenomena shed new insight on internal QC laser processes and may themselves lead to further improvements in device performance.

\clearpage

\cleardoublepage

%%
%% ACKNOWLEDGEMENTS
%%
\addcontentsline{toc}{section}{Acknowledgements}
\begin{center}
    \Large \textbf{Acknowledgements}
\end{center}

One quickly learns in graduate school that progress in research is seldom accomplished alone.  The work I present in this thesis is no different, and I am indebted to a great many people for their help, support, friendship, and, patience over the last five years.

My adviser, Professor Claire Gmachl, is most certainly at the top of the list. She is everything you could ever want in a graduate adviser, and a whole lot more.  It's taken me the better part of five years to learn that, when we have a disagreement over science, she always ends up being right; the important part was that she let me discover this on my own.

Both Professor Andrew Houck and Professor Gerard Wysocki were kind enough to give up more than one weekend to review this thesis, and I am extremely grateful for their time and valuable input.

One of the greatest advantages of working in Prof. Gmachl's group is the tremendous opportunity provided for collaborative research.  I have been fortunate to work with many different groups on several projects, and each of these collaborations has contributed meaningfully to this dissertation.

Before his move to Michigan, Professor Stephen Forrest was kind enough to welcome me into his research group and lab.  Using his MBE to grow QC lasers was an invaluable experience.  And for their time and patience, I thank Dr. Shashank Agashe, Stephane Kena-Cohen, and especially Kuen-Ting Shiu in teaching me the art of MBE.

My work at Princeton has benefited from several MBE and MOCVD growth collaborators.  The II--VI project would have been impossible without the assistance of Professor Maria Tamargo and her group.  Professor Aidong Shen, Dr. Hong Lu, and especially William Charles all played significant roles in our progress so far on II--VI QC structures.

I am also deeply indebted to Mary Fong, Jen-Yu Fan, and Xiaojun Wang at AdTech Optics for all of their assistance and support.  Not only did they provide excellent QC growth, but they welcomed me as an intern during the summer of 2007. They made it an amazing experience for me, and I'm very appreciative.

Other key collaborators include Professor Fow-Sen Choa and his group at University of Maryland--Baltimore County, Dr. Jerry Meyer and his group at the Naval Research Lab, and Dr. Fred Towner at Maxion Technologies.  I'd also like to thank Dr. Igor Trofimov, always an excellent source of advice for processing and fab techniques and tricks.

I am deeply grateful to Prof. Jacob Khurgin and Dr. Yamac Dikmelik at Johns Hopkins University for a number of extremely fruitful discussions over the last few years.  The success of our research team would not be without their valuable membership.

I would also like to thank my fellow members and visitors of the Mid-Infrared Photonics Group at Princeton University: Professor Dan Wasserman, Dr. Jian-Zhang Chen, Dr. Jianxin Chen, Dr. Anna Michel, Dr. Zhijun Liu, Dr. Afusat Dirisu, Fatima Toor, Matthew Escarra, Richard Cendejas, Ekua Bentil, Yu Yao, Peter Liu, Elvis Mujagi\'{c}, Aleksander W\'{o}jcik, Alexander Benz, Abhishek Agrawal, and Tiffany Ko. Working with everyone was a pleasure and I wish you all the best of luck.

Two fellow members of my research group deserve special thanks.  Dr. Scott Howard was there from day one when I couldn't tell a lock-in amplifier from a pulse generator, and he has since become a great friend.  I also want to thank Dr. Anthony Hoffman, my lab mate, office mate, roommate, and the best of friends.  Scott and Anthony, along with Professor Gene Slowinski, all fellow co-founders of Primis Technologies LLC, deserve special recognition for picking up the slack while I've been racing to complete this dissertation.

I also want to acknowledge the support of the National Science Foundation in making my graduate work possible.  Both through the NSF Graduate Research Fellowship and NSF's funding of the MIRTHE center, the financial assistance has been invaluable.  I am also grateful for the support of the Princeton University Wallace Fellowship during my final year of graduate work.

Finally, I want to thank my family: my mother Marlinda, father Kyle, grandma Jeanne, grandpa Melvin, sister MaKayla, brother-in-law Philip, and nephew Merrick.   For all the love and support they've given me, for allowing me to spend five years on the other side of the country to pursue my graduate work, and for all of their inspiration and encouragement along the way, I am eternally grateful.

\clearpage

\cleardoublepage


%%
%% DEDICATION PAGE
%%
%\thispagestyle{empty}
%\begin{flushright}
%    to my family
%\end{flushright}
%\clearpage
%\cleardoublepage 